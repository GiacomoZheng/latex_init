\author{Renpeng Zheng}
\address{School of Mathematical Sciences \\University of Nottingham \\Nottingham \\NG7 2RD \\United Kingdom}
\email{Renpeng.Zheng@Nottingham.ac.uk}
% =================================
% Packages
% =================================
\usepackage[utf8]{inputenc}
\DeclareUnicodeCharacter{211A}{$\mathbb{Q}$}
\DeclareUnicodeCharacter{211D}{$\mathbb{R}$}
\usepackage[T1]{fontenc}
% \usepackage{libertine}
% \usepackage[libertine]{newtxmath}
% \usepackage{newtx}

\usepackage[bb=libus]{mathalpha}
\usepackage{amsmath,latexsym,amsfonts}
\usepackage{amssymb}


\usepackage{CJKutf8}
\newcommand{\zh}[1]{\begin{CJK*}{UTF8}{gbsn}#1\end{CJK*}}
\usepackage[french, english]{babel}
\usepackage[strict=true]{csquotes}
\babelprovide{chinese}
\setlocalecaption{chinese}{abstract}{\zh{摘要}}
\newenvironment{abstractZH}{
  \begin{otherlanguage}{chinese}
    \begin{abstract}
      \begin{CJK*}{UTF8}{gbsn}
}{
      \end{CJK*}
    \end{abstract}
  \end{otherlanguage}
}
% ---------------------------------
\usepackage[x11names, svgnames, rgb]{xcolor}
\usepackage[
  linktoc=page,
  colorlinks,
  citecolor=gray,
]{hyperref}
% ---------------------------------
\usepackage{mathtools}
\usepackage{amsthm}
\newtheorem{theorem}{Theorem}[section]
\newtheorem*{theorem*}{Theorem}
\newtheorem{proposition}[theorem]{Proposition}
\newtheorem{corollary}[theorem]{Corollary}
\newtheorem{lemma}[theorem]{Lemma}
\newtheorem{construction}[theorem]{Construction}
{
  \theoremstyle{definition}
  \newtheorem{definition}[theorem]{Definition}
  \newtheorem{example}[theorem]{Example}
  \newtheorem{question}[theorem]{Question}
  \newtheorem{conjecture}[theorem]{Conjecture}
  \newtheorem*{convention}{Convention}
  \newtheorem{lemdefn}[theorem]{Lemma-Definition}
}
{
  \theoremstyle{remark}
  \newtheorem{remark}[theorem]{Remark}
  \newtheorem*{remark*}{Remark}
  \newtheorem{claim}[theorem]{Claim}
}
\newenvironment{sketchproof}{%
  \renewcommand{\proofname}{Proof (sketch)}\proof
}{\endproof}
% ---------------------------------
\usepackage{mdframed}
\newmdenv[
    linecolor=black,
    linewidth=0.1pt,
    topline=false,
    bottomline=false,
    rightline=false
]{inline}
% ---------------------------------
\usepackage{cleveref}
\crefname{claim}{claim}{claims}
\crefname{conjecture}{conjecture}{conjectures}
\crefname{question}{question}{questions}
\crefname{lemdefn}{lemma-definition}{lemma-definitions}
\Crefname{lemdefn}{Lemma-Definition}{Lemma-Definitions}
% ---------------------------------
\usepackage[
  style=alphabetic,
]{biblatex}
\DeclareLanguageMapping{chinese}{english}
% ---------------------------------
\usepackage{tikz}
\usetikzlibrary{cd}
\usetikzlibrary{arrows,shapes}
\usetikzlibrary{calc}
\usepackage[extract=no]{memoize}
\mmzset{auto = {mytikzcd}{memoize, verbatim}}
\newenvironment{mytikzcd}{\begingroup
    \catcode`\[=12 \catcode`\]=12
  \begin{tikzcd}}{\end{tikzcd}\endgroup}
% ---------------------------------
\usepackage{comment}
\specialcomment{wait}{\begingroup\color{blue}}{\endgroup}
\specialcomment{warn}{\begingroup\color{red}}{\endgroup}
% ---------------------------------
% =================================
% customise
% =================================
% ---------------------------------
%% for \over
\makeatletter
\let\over\@@over
\makeatother
% ---------------------------------
%% for (), [] and \{\}
\def\oparen{\mathopen{}\left\lparen}
\def\cparen{\right\rparen\mathclose{}} % ()
\def\obrack{\mathopen{}\left\lbrack}
\def\cbrack{\right\rbrack\mathclose{}} % []
{
  \catcode`\(=13 \gdef({\oparen} \catcode`\)=13 \gdef){\cparen}
  \catcode`\[=13 \gdef[{\obrack} \catcode`\]=13 \gdef]{\cbrack}
}
\everydisplay\expandafter{\the\everydisplay
  \def\{{\mathopen{}\left\lbrace} \def\}{\right\rbrace\mathclose{}} \def\mid{\,\middle|\,} % \{\} and |
  \catcode`\(=13 \catcode`\)=13 \catcode`\[=13 \catcode`\]=13
}
% for \langle and \rangle <>
\newcommand{\<}{\left\langle} \renewcommand{\>}{\right\rangle}
% ---------------------------------
% % for pairs
\DeclarePairedDelimiter\ceil{\lceil}{\rceil}
\DeclarePairedDelimiter\floor{\lfloor}{\rfloor}
% ---------------------------------
%% for amsmath tilte
\let\origmaketitle\maketitle
\def\maketitle{
  \begingroup
  \def\uppercasenonmath##1{} % this disables uppercasing title
  \let\MakeUppercase\relax % this disables uppercasing authors
  \origmaketitle
  \endgroup
}
% ---------------------------------
% Table of Contents
\setcounter{tocdepth}{3}
\makeatletter
\renewcommand{\tocsection}[3]{%
  \indentlabel{\@ifnotempty{#2}{\ignorespaces#1 #2.~}}\bfseries#3}
\renewcommand{\tocsubsection}[3]{\color{black!60}%
  \indentlabel{\@ifnotempty{#2}{\ignorespaces#1 #2.~}}#3}
\renewcommand{\tocsubsubsection}[3]{\color{black!60} 
  \indentlabel{\@ifnotempty{#2}{\ignorespaces#1 \phantom{1.1.}~}}#3}

\def\@tocline#1#2#3#4#5#6#7{\relax
  \ifnum #1>\c@tocdepth % then omit
  \else
    \def\author##1{\newline\textsc{##1}}%
    \par \addpenalty\@secpenalty\addvspace{#2}%
    \begingroup
      \hyphenpenalty\@M
      \@ifempty{#4}{%
        \@tempdima\csname r@tocindent\number#1\endcsname\relax
      }{%
        \@tempdima#4\relax
      }%
      \parindent\z@ \leftskip#3\relax \advance\leftskip\@tempdima\relax
      \rightskip\@pnumwidth plus.2\hsize \parfillskip-\@pnumwidth
      #5\leavevmode\hskip-\@tempdima #6\nobreak\relax
      ~ \dotfill  % filling the blank
      \hbox to\@pnumwidth{\@tocpagenum{#7}}\par
      \nobreak
    \endgroup
  \fi
}
\def\l@subsection{\@tocline{2}{0pt}{1pc}{5pc}{}}
\def\l@subsubsection{\@tocline{3}{0pt}{1pc}{5pc}{}}
\makeatother
% ---------------------------------
% % for \limts
\let\oldbigoplus\bigoplus
\renewcommand{\bigoplus}{\oldbigoplus\limits}
\let\oldbigotimes\bigotimes
\renewcommand{\bigotimes}{\oldbigotimes\limits}
\let\oldbigcap\bigcap
\renewcommand{\bigcap}{\oldbigcap\limits}
\let\oldbigcup\bigcup
\renewcommand{\bigcup}{\oldbigcup\limits}
\let\oldbigsqcup\bigsqcup
\renewcommand{\bigsqcup}{\oldbigsqcup\limits}
\let\oldlim\lim
\renewcommand{\lim}{\oldlim\limits}
\let\oldlimsup\limsup
\renewcommand{\limsup}{\oldlimsup\limits}
\let\oldliminf\liminf
\renewcommand{\liminf}{\oldliminf\limits}
\DeclareMathOperator*{\wlim}{\textrm{wlim}}
\let\oldsup\sup
\renewcommand{\sup}{\oldsup\limits}
\let\oldinf\inf
\renewcommand{\inf}{\oldinf\limits}
\let\oldsum\sum
\renewcommand{\sum}{\oldsum\limits}
\let\oldprod\prod
\renewcommand{\prod}{\oldprod\limits}
% \let\oldint\int
% \renewcommand{\int}{\oldint\limits}
% ---------------------------------
% Delete some command
\let\olddiv\div
\renewcommand{\div}{\bug}
% ---------------------------------
% redef some command
\renewcommand{\emptyset}{\varnothing}
\renewcommand{\epsilon}{\varepsilon}
\renewcommand{\hat}{\widehat}
% ---------------------------------
\newcommand{\surj}{\twoheadrightarrow}
\newcommand{\inj}{\hookrightarrow}
\newcommand{\cl}{\overline}

\newcommand{\Proj}{\mathrm{Proj}}
\newcommand{\Spec}{\mathrm{Spec}}

\newcommand{\lcm}{\mathrm{lcm}}

\newcommand{\supp}{\mathrm{supp}}
\newcommand{\codim}{\mathrm{codim}}
\newcommand{\card}{\#}

\newcommand{\pr}{\mathrm{pr}}

\newcommand{\LHS}{\mathrm{LHS}}
\newcommand{\RHS}{\mathrm{RHS}}
\newcommand{\Hom}{\mathrm{Hom}}
\newcommand{\End}{\mathrm{End}}
\newcommand{\Aut}{\mathrm{Aut}}

\newcommand{\alggrp}{\mathrm{alg.grp}}
\newcommand{\grp}{\mathrm{grp}}

\newcommand{\Gm}{\mathbb{G}_\mathrm{m}}
\newcommand{\Ga}{\mathbb{G}_\mathrm{a}}

\newcommand{\rank}{\mathrm{rank}}

\newcommand{\dd}{\mathrm{d}}

% I am not sure 
\newcommand{\interior}{\mathrm{int}}
\newcommand{\res}[1]{\left.#1\right|}